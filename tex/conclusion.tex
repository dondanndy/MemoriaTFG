Para determinar la precisión en el posicionamiento de un kit de sensores de Ultra-wide Band se ha procedido a tomar varias medidas con la ayuda de un robot capaz de desplazarse de forma autónoma en dos escenarios, uno libre de objetos y un segundo con una superficie mayor y con obstáculos que pudieran impedir en ciertas ocasiones el correcto desempeño del posicionamiento del kit utilizado.

Además de las distintas localizaciones, en cada uno de los dos escenarios se han recogido datos con distintas configuraciones de balizas, de tal forma que ha sido posible determinar el efecto de su colocación en el posicionamiento del objetivo.

Las medidas en los dos escenarios han arrojado resultados similares: una distribución uniforme de balizas proporciona mejores resultados fruto de dar opción al kit de utilizar sensores más cercanos para el posicionamiento del objetivo.

Así, para maximizar el rendimiento de este kit no solo se deberán usar un número mayor de balizas, si no que será necesario un estudio que permita obtener una configuración en la zona de interés tal que los objetivos a posicionar se encuentren a la menor distancia posible de al menos 4 balizas.

En ninguno de los dos escenarios contemplados en este trabajo se han obtenido valores medios de error en el posicionamiento mayores a 30cm, teniendo en cuenta además que en todos ellos, además del error de precisión dado por la tecnología de UWB, se añadía el posible error de posicionamiento del robot.
Es decir, es posible encontrar en superficies del orden de decenas de metros cuadrados, errores medios de, al menos, dos órdenes de magnitud menores.

Estos resultados abren la puerta a su empleo en el posicionamiento de robots teniendo en cuenta que dicho error entrará en la mayoría de casos en sus propias dimensiones o con humanos, siendo 30cm apenas dar un paso.

En superficies aún mayores que las tratadas se deberían obtener unos resultados parecidos siempre con una colocación de un número mayor de balizas respetando las condiciones de homogeneidad en las distancias ya discutida.

Aun así, el problema del posicionamiento local no está completamente resuelto.
Los resultados expuestos en este trabajo son valores medios obtenidos a partir de numerosas medidas de un kit de UWB, pero en algunas de las aplicaciones donde se necesite realizar el posicionamiento de forma veloz la naturaleza aleatoria de las señales empleadas pueden provocar un pobre rendimiento de estas técnicas.

Obstáculos dinámicos también juegan un gran papel en la precisión del posicionamiento local.
En el caso del UWB la visión directa entre los sensores es crítica para su correcto funcionamiento y la posición de las balizas debe contemplarla, por lo que escenarios con obstáculos impredecibles evitarán su correcto funcionamiento.

La proliferación de la tecnología UWB da una nueva alternativa a los sistemas de posicionamiento local ya establecidos y discutidos en este trabajo.
Tratando de evitar en la medida de lo posible los casos donde aparecen sus puntos débiles, proporciona una precisión y robustez muy superior a sus alternativas, por lo que se presenta en estos momentos como una de la mejores opciones en este ámbito.