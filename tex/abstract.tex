Global positioning systems, based on satellite signals, are not able to provide a high level of precision in indoor environments due to the lack of direct line of sight and the attenuation that this entails.
In these scenarios it is necessary to use different techniques in order to achieve a similar positioning.

Among the technologies used, Ultra-wide Band, which uses pulses of at least 500MHz bandwidth, has been gaining traction as it is lowly intrusive and highly effective in a saturated electromagnetic spectrum giving a precision in the order of centimeters depending on the scenario.

In this work two locations are chosen, with ideal and non-ideal conditions, to evaluate the precision of the DecaWave MDEK1001 commercial kit using a robot.
Testing various setups in both scenarios, the performance of the kit in several situations was compared to determine how obstacles affect the position estimations it provides.

In both scenarios, it can be seen that the anchors setup is the main factor in obtaining the least possible positioning error.
The smallest errors obtained were in the order of 20~cm, with slightly higher values at the non-ideal location.