En este trabajo se trata de evaluar la capacidad de la tecnología Ultra-wide Band para las tareas de posicionamiento local.
Para este acometido es fundamental tener en cuenta el entorno en el que se intenta conseguir dicho posicionamiento y la disposición de todo el entramado de sensores para encontrar la configuración óptima.

Por ello, los objetivos del trabajo son
\begin{itemize}
    \item Evaluar la precisión de los sensores de Ultra-wide Band en un entorno favorable, de tal manera que es posible encontrar los límites de una forma fácilmente parametrizable y sin incluir fuentes adicionales de error.
    \item Evaluar de nuevo la precisión de los sensores en un segundo escenario menos favorable, con obstáculos que puedan interferir en las tareas de posicionamiento y así poder comprobar cómo afectan estas nuevas variables a los errores en la posición obtenida.
    \item Comparar el desempeño de la tecnología en ambos escenarios, y las discrepancias frente a los valores teóricos.
    \item En ambos casos, realizar tomas de medidas con distintas configuraciones de los sensores, y así poder evaluar cuáles de ellas arrojan mejores resultados y así encontrar el mejor equilibrio entre su desempeño y su complejidad.
\end{itemize}