En este apéndice figura el código utilizado para construir los nodos encargados del movimiento del robot y de la recogida de datos de posicionamiento, tanto de odometría como de los sensores de UWB.

Para todo el código se preparó un archivo de base con funciones comunes.
En este archivo figura el todo el código excepto la generación de las posiciones de cada trayectoria, para las que se programó una función en un archivo adicional para cada una, compilando un nodo para cada caso.

El archivo común, llamado \texttt{base.h}, contiene el siguiente código:
\lstset{language=C++,
	extendedchars=true,
	keywordstyle=\bfseries\color{NavyBlue},
	stringstyle=\color{OliveGreen},
	commentstyle=\color{gray},
	numbers=left,
	numberstyle=\tiny,
	basicstyle=\footnotesize,
	breaklines=true,
	tabsize=2}

\begin{lstlisting}
    #include <string>
    #include <vector>
    #include <array>
    #include <fstream>
    #include <chrono>
    #include <thread>
    #include <ros/ros.h>
    #include <std_msgs/Float64.h>
    #include <move_base_msgs/MoveBaseAction.h>
    #include <geometry_msgs/PoseWithCovarianceStamped.h>
    #include <actionlib/client/simple_action_client.h>
    #include <tf/transform_datatypes.h>
    #include <tf/transform_listener.h>
    
    typedef geometry_msgs::Pose Pose;
    typedef geometry_msgs::PoseWithCovarianceStamped PoseCov;
    
    const std::string ruta("/home/tb2b/catkin_ws/src/"); //Ruta en el ordenador del robot
    
    const unsigned int MAX_PTS_SENSOR = 50; //Puntos a recoger de los sensores.
    const unsigned int MAX_INT = 3; // Intentos en caso de que el robot no pueda llegar al objetivo
    const unsigned int REP = 3; //Repeticiones de cada trayectoria.
    const bool SALIDA_TF = false;
    
    // Posicion del robot.
    Pose actual_pose;
    boost::array<double,36> actual_pose_covariance;
    
    std::vector<Pose> path();
    std::string get_tag_name();
    
    std::string get_sensor_data(){
        std::ifstream consola;
        std::ofstream file; 
        std::string linea;
        unsigned int num_datos = 0;
    
        while (num_datos < MAX_PTS_SENSOR){
            
            consola.open("/dev/ttyACM0");
    
            std::getline(consola,linea);
    
            if (linea.size() > 115 && linea.size() < 180) {
                // A veces se vuelve un poco loco al leer asi que no guardaremos los datos que 
                // por estar con menos caracteres no nos sirven.
                
                std::this_thread::sleep_for(std::chrono::milliseconds(100));
    
                file.open(ruta + "turtle_tfg/src/datos/sensor/" + tag + ".txt", std::ios_base::app);
                file << linea << "\n";
                file.close();
            
                num_datos++;
            }
            
            consola.close();
        }
    
            //TODO
        return("0.0\n");
    }
    
    std::string get_date(){
        /*
            Devuelve fecha y hora.
         */
    
        auto now = std::chrono::system_clock::to_time_t(std::chrono::system_clock::now());
    
        //Fecha a texto
        char text[50];
        std::strftime(text, sizeof(text), "%Y-%m-%d--%Hh%Mm", std::localtime(&now));
    
        std::string string_return(text); // Devolvemos std::string en vez de char*
        return(string_return);
    }
    
    void log_position(const std::string& file_name, const Pose& pos){
        /* 
            Escribe, en el archivo file_name, la posicion del robot y de los sensores.    
        */
        std::string log_pos; //Posicion a escribir en el archivo
    
        log_pos = std::to_string(pos.position.x) + "\t" + std::to_string(pos.position.y) + "\t"; // Log de la posicion teorica
    
        //Posicion real, tf
        if (SALIDA_TF){
        // En un principio no vamos a usar tf para conseguir la posicion asi que lo escondemos detras de un bool en false.
    
            tf::TransformListener listener;
            tf::StampedTransform trans;  
            
            bool dato_OK = true;
            while(dato_OK){
                //Reintentamos hasta que damos con el topic.
                try{
                    listener.lookupTransform("map", "base_link", ros::Time(), trans);
        
                    std::cout << "Posicion tf -----" << std::endl;
        
                    std::cout << trans.getOrigin().getX() << "," << trans.getOrigin().getY() << std::endl;
                    
                    double roll, pitch, yaw;
                    trans.getBasis().getRPY(roll, pitch, yaw);
                    std::cout << "Yaw - " << yaw << std::endl;
                    dato_OK = false; //Salimos.
                
                }
                catch(tf::TransformException& ex) {
                    //std::cout << "Buscando transformacion \r";
                }
        }}
    
        //amcl_pose
    
        std::cout << "Posicion amcl -----" << std::endl;
        std::cout << actual_pose.position.x << "," << actual_pose.position.y << std::endl;
        std::cout << actual_pose_covariance[0] << "," << actual_pose_covariance[7] << std::endl;
    
        std::cout << "--------------------" << std::endl;
    
        log_pos = log_pos
                  + std::to_string(actual_pose.position.x) + "\t"
                  + std::to_string(actual_pose_covariance[0]) + "\t"
                  + std::to_string(actual_pose.position.y) + "\t"
                  + std::to_string(actual_pose_covariance[7]) + "\t"
                  + get_sensor_data(); // Log de la posicion de amcl
    
        // Al archivo
        std::ofstream file(file_name, std::ios_base::app);
        file << log_pos;
        file.close();
    }
    
    bool move_to_goal(const Pose& pos){
        /*
            Mueve el robot hacia una posicion en el mapa.
            Devuelve si el resultado ha sido exitoso.
        */
        
        actionlib::SimpleActionClient<move_base_msgs::MoveBaseAction> ac("move_base", true);
        while(!ac.waitForServer(ros::Duration(5.0))){
            ROS_INFO("Iniciando server");
        }
        
        move_base_msgs::MoveBaseGoal goal;
        
        goal.target_pose.header.frame_id = "map";
        goal.target_pose.header.stamp = ros::Time::now();
        goal.target_pose.pose = pos;
    
        //Movimiento
        ac.sendGoal(goal);
        ac.waitForResult();
        return(ac.getState() == actionlib::SimpleClientGoalState::SUCCEEDED);
    }
    
    
    void update_position(const PoseCov& position){
        actual_pose = position.pose.pose;
        actual_pose_covariance = position.pose.covariance;
    }
    
    int main(int argc, char** argv){
    
        ros::init(argc, argv, "map_navigation");
        ros::NodeHandle n;
        ros::Subscriber sub = n.subscribe("amcl_pose",500, update_position);
        ros::spinOnce();
        bool suc;
    
        std::string date = get_date();
    
        std::string file = ruta + "/turtle_tfg/datos/" + get_date() + ".txt"; // Nombre del archivo para guardar los puntos.
    
        auto posiciones = path(); // Vector con las posiciones de la trayectoria.
        
        for (int k=1; k <= REP; k++){
            file = ruta + "turtle_tfg/src/datos/" 
                   + get_tag_name() + "-" + date + "-" + std::to_string(k) + ".txt"; 
    
            for (int i=0; i < posiciones.size(); i++){
                suc = move_to_goal(posiciones[i]);
                if (!suc){
                    ROS_INFO("El robot no ha llegado a su destino, reintentando");
                    for (int j=0; j < MAX_INT; j++){
                        suc = move_to_goal(posiciones[i]);
                        if (suc) break;
                    }
                }
            ros::spinOnce();
    
            std::string tag = get_tag_name() + std::to_string(i) + date + "-" + std::to_string(k); //Tag de cada punto.
            log_position(file, posiciones[i], tag);
            };
        };
    
        ROS_INFO("Final de la trayectoria.");
        return 0;
    }

\end{lstlisting}

Junto con el archivo base se escribió un archivo adicional para cada una de las trayectorias con la definición de la función \texttt{path()}, que proporciona los puntos a recorrer por el robot.

En el caso de la trayectoria en espiral tal y como se indica en la Figura \ref{fig:espiral} del laboratorio dicho archivo fue

\begin{lstlisting}
#include <base.h>
void log_position(const Pose& pos);
bool move_to_goal(const Pose& pos);
void update_position(const PoseCov& position);

std::string get_tag_name(){
    return("ESP");
}

std::vector<Pose> path() {
        std::vector<Pose> ruta;
        Pose position;

        const int MAX = 2;

        for(int i=0; i <= MAX; i++){

            int x = i;
            int y = i;

            position.position.x = x;
            position.position.y = y;

            position.orientation = tf::createQuaternionMsgFromYaw(0); //0 grados

            ruta.push_back(position);

            // Trayecto hacia abajo
            for(int j=i; j>-i; j--){
                y -= 1;

                position.position.x = x;
                position.position.y = y;

                position.orientation = tf::createQuaternionMsgFromYaw(0); //0 grados

                ruta.push_back(position);
            }

            // Trayecto hacia la izquierda
            for(int j=i; j>-i; j--){
                x -= 1;

                position.position.x = x;
                position.position.y = y;

                position.orientation = tf::createQuaternionMsgFromYaw(0); //0 grados

                ruta.push_back(position);
            }

            // Trayecto hacia arriba
            for(int j=i; j>-i; j--){
                y += 1;

                position.position.x = x;
                position.position.y = y;

                position.orientation = tf::createQuaternionMsgFromYaw(0); //0 grados

                ruta.push_back(position);
            }

            // Trayecto hacia la derecha
            for(int j=i; j>-i+1; j--){
                x += 1;

                position.position.x = x;
                position.position.y = y;

                position.orientation = tf::createQuaternionMsgFromYaw(0); //0 grados

                ruta.push_back(position);
            }
        }
        return(ruta);
}
\end{lstlisting}

En el caso de la trayectoria en vertical que se indica en la Figura \ref{fig:vertical} archivo contenía el siguiente código:

\begin{lstlisting}
#include "base.h"

std::string get_tag_name(){
    return("VERT");
}

std::vector<Pose> path(){
    std::vector<Pose> ruta;
    Pose position;

    const int ANCHO = 2;
    const int ALTO = 3;

    bool orient = false; //Comenzamos hacia abajo.

    for (int i=ANCHO; i >= -ANCHO; i--){
        for (int j=ALTO; j >= -ALTO; j--){
            if (orient){
                position.position.x = i;
                position.position.y = -j;

                position.orientation = tf::createQuaternionMsgFromYaw(0); //0 grados
            } else {
                position.position.x = i;
                position.position.y = j;

                position.orientation = tf::createQuaternionMsgFromYaw(0); //0 grados
            }
        ruta.push_back(position);
        }
    orient = !orient;
    }
    return(ruta);
}
\end{lstlisting}

La tercera y última trayectoria hacía que el robot recorriese todo el laboratorio de forma aleatoria.
Se planteó que en cada trayectoria el orden cambiase, como se indica en el código que la genera:
\begin{lstlisting}
#include "base.h"
// Trayectoria que recorre el laboratorio de forma aleatoria.

std::string get_tag_name(){
    return("RAND");
}

std::vector<Pose> path(){
    std::vector<Pose> ruta;
    Pose position;

    const unsigned int ANCHO = 2;
    const unsigned int ALTO = 3;

    //Distribuciones pseudo-aleatorias
    std::random_device rd;
    std::mt19937 gen(rd());

    // Generamos todos los datos posibles.
    for (int i = -ANCHO; i <= ANCHO; i++){
        for (int j = -ALTO; j <= ALTO; j++){
            position.position.x = i;
            position.position.y = j;
            position.orientation = tf::createQuaternionMsgFromYaw(0);

            ruta.push_back(position);
        }
    }

    // Mezcla aleatoria del vector.
    std::shuffle(ruta.begin(), ruta.end(), g);

    return(ruta);
}    
\end{lstlisting}

La última trayectoria fue la utilizada en el edificio de la Facultad de Física, con el siguiente código:
\begin{lstlisting}
#include <base.h>

void log_position(const Pose& pos);
bool move_to_goal(const Pose& pos);
void update_position(const PoseCov& position);

std::string get_tag_name(){
    return("FIS");
}

std::vector<Pose> path() {
        std::vector<Pose> ruta;
        Pose position;

        float min_x = 0.7;
        float min_y = -0.5;

        float max_x = 10.0;
        float max_y = 10.0;

        const int MAX_PTS_TRAY = 3;

        //Primera parte
        for (int n=0; n <= MAX_PTS_TRAY; n++){
            position.position.x = min_x + (max_x - min_x)*n/(MAX_PTS_TRAY + 1);
            position.position.y = min_y ;
            
            position.orientation = tf::createQuaternionMsgFromYaw(0); //0 grados
            ruta.push_back(position);
        }


        //Segunda parte
        for (int n=0; n <= MAX_PTS_TRAY; n++){
            position.position.x = max_x ;
            position.position.y = min_y + (max_y - min_y)*n/(MAX_PTS_TRAY + 1);
            
            position.orientation = tf::createQuaternionMsgFromYaw(0); //0 grados
            ruta.push_back(position);
        }

        //Tercera parte
        for (int n=0; n <= MAX_PTS_TRAY; n++){
            position.position.x = max_x - (max_x - min_x)*n/(MAX_PTS_TRAY + 1);
            position.position.y = max_y ;
            
            position.orientation = tf::createQuaternionMsgFromYaw(0); //0 grados
            ruta.push_back(position);
        }

        //Cuarta parte
        for (int n=0; n <= MAX_PTS_TRAY; n++){
            position.position.x = min_x ;
            position.position.y = max_y - (max_y - min_y)*n/(MAX_PTS_TRAY + 1);
            
            position.orientation = tf::createQuaternionMsgFromYaw(0); //0 grados
            ruta.push_back(position);
        }

        return(ruta);
}
\end{lstlisting}
