\textbf{Los sistemas de posicionamiento global, basados en la señal de satélites, no presentan una gran precisión en ambientes interiores debido a la falta de visión directa y la atenuación que conlleva, por lo que en estos escenarios es necesario emplear técnicas distintas para lograr un posicionamiento similar.}

Dentro de las distintas tecnologías empleadas el Ultra-wide Band (UWB), que usa señales de al menos 500MHz de ancho de banda, está ganando poco a poco popularidad por ser una tecnología poco intrusiva y efectiva en un espectro electromagnético muy poblado, con una precisión en el orden del centímetro, pero muy dependiente del entorno donde se emplea.

Así, en este trabajo se plantean dos entornos para evaluar, en condiciones de uso ideales y no ideales, la precisión del kit comercial de UWB DecaWave MEDK1001 con la ayuda de un robot.
Con varias configuraciones de los sensores en ambos escenarios, se comparó el desempeño del kit en diversas situaciones para determinar cómo afectan los obstáculos a las estimaciones de posición que proporciona.

\textbf{Aunque la falta de visión directa tiene un gran impacto en el rendimiento del kit, evitar los casos donde se da con una correcta disposición de las balizas, que tiene una importancia crítica a la ahora de obtener el menor error en el posicionamiento obteniendo.}