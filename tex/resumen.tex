Debido a las limitaciones del uso de sistemas de posicionamiento global en interiores aparecen soluciones parecidas, pero adaptadas a estos ambientes con dimensiones menores y ambientes diferentes.
Dentro de las distintas tecnologías empleadas el Ultra-wide Band, que usa señales de al menos 500MHz de ancho de banda, está ganando poco a poco popularidad por ser una tecnología poco intrusiva y efectiva en un espectro electromagnético muy poblado, con una precisión en el orden del centímetro, pero muy dependiente del entorno donde se emplea.

Así, en este trabajo se plantean dos localizaciones para evaluar, en condiciones de uso ideales y no ideales, la precisión del kit comercial DecaWave MEDK1001 con la ayuda de un robot.
Con varias configuraciones de los sensores en ambos escenarios, se comparó el desempeño del kit en diversas situaciones para determinar cómo afectan los obstáculos a las estimaciones de posición que proporciona.