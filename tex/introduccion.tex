\subsection{Introduccion}

El éxito de los sistemas de posicionamiento global como el GPS ha iniciado en los últimos años el desarrollo de tecnologías y experiencias basadas en el control y el seguimiento de la posición de objetos o personas.

En ambientes exteriores y abiertos, la precisión de este tipo de sistemas, con el equipamiento adecuado, puede llegar a ser de centímetros.
No es el caso en situaciones interiores, donde la nula línea de visión directa entre los medidores y los satélites hacen que esta precisión sea imposible, por lo que su uso, en general, es inviable.

Debido a estos inconvenientes han aparecido nuevas técnicas para conseguir un posicionamiento similar, pero de forma local.
Aunque su objetivo es el mismo, los entornos interiores requieren una precisión mayor y tienen, en general, un espectro electromagnético más poblado, por lo que se deben emplear tecnologías distintas compatibles con estos escenarios.

\begin{figure}[H]
    \centering
    \def\svgwidth{1.1\linewidth}
    \input{./fig/general.pdf_tex}
	\caption{Esquema general de un sistema de posicionamiento local.}
    \label{fig:Gen}
\end{figure}

% Con el mismo resultado como objetivo estas tecnologías tienen serias diferencias debido a las circunstancias para las que se desarrollan: entre ellas se encuentran principalmente la precisión requerida, mucho mayor en interiores, o la necesidad de lidiar con un espectro electromagnético más poblado.

A diferencia del caso del posicionamiento global, existen dos tipos de algoritmos: unos basados en argumentos geométricos a partir de emisores con posición conocida, y otros consistentes en un mapeado previo de la zona de interés.
En ambos casos se utilizan balizas de posición conocida, como se indica en el esquema de la Figura~\ref{fig:Gen}.
Estas técnicas y las tecnologías sobre las que se implementan son las principales diferencias entre las alternativas comerciales disponibles.

La proliferación de la comunicación inalámbrica ha provocado una saturación en el espectro electromagnético, de tal forma que nuevos usos que puedan emplear este tipo de ondas se vean seriamente afectados por interferencias en la frecuencias en uso.

Aunque su planteamiento se remonta varias decenas de años en el tiempo, es cada vez más habitual la mención al Ultra-wide Band (UWB), una tecnología que usa un espectro de ondas de radio con un ancho de banda superior al 20\% de la frecuencia central o mayor que 500 MHz.

Su gran ancho de banda permite la emisión de pulsos muy cortos en el tiempo y de baja energía, lo que hace que su implantación tenga una alta eficiencia.
Además no requiere diseños complejos para la emisión ni para la recepción de pulsos, con lo que el equipamiento necesario para su uso tendrá en general un bajo coste.

La baja tolerancia que proporciona un uso tan amplio de frecuencias a las interferencias --a diferencia del Bluetooth o Wi-Fi, con canales con ancho de banda mucho más pequeño-- la hacen muy atractiva para su uso masivo teniendo en cuenta que es posible asegurar un buen funcionamiento independientemente del uso simultáneo de otras tecnologías.

El estudio de la precisión disponible del UWB es el principal foco de este trabajo.
La baja potencia de los pulsos usados para el posicionamiento hace que la presencia de obstáculos sea su principal punto débil, con una capacidad de penetración muy baja.

En escenarios con una configuración en la que el objetivo del posicionamiento no tenga línea visión directa de las balizas no será posible tener mediciones excesivamente precisas, y determinar los límites en dichos casos será crítico para evaluar el uso del UWB o de alguna de sus alternativas.

Para automatizar la toma de datos en la medida de lo posible se ha usado un robot en el que colocar el sensor de UWB a posicionar, de tal forma que ha sido posible una toma de datos con un volumen mayor a comparación de una metodología manual.

El robot utilizado ha sido el Turtlebot 2, un pequeño robot con fines educativos que permite el uso de cámaras y sensores láser para labores de movimiento autónomo, de tal forma que permite un movimiento total en un entorno controlado.
Aunque es posible una navegación libre a partir de los sensores mencionados, para este trabajo se creó un mapa de los dos escenarios en los que se produjo la toma de medidas para que su posicionamiento local fuese lo más preciso posible.

Dicho robot fue empleado en dos escenarios para poder comparar el desempeño de los sensores de UWB en dos situaciones distintas.
Así, en la elección de los dos escenarios se primó la posibilidad de comparar una situación ideal, con visión directa y distancias cortas; y una situación más parecida a un uso habitual.
En este último escenario se contemplaba que, en ocasiones, pueda haber obstáculos que dificulten el traslado de la señal de UWB.

La primera toma de medidas se llevó a cabo en uno de los laboratorios de los Institutos Universitarios de Investigación en el que se disponía de una superficie amplia y libre; la segunda localización tuvo lugar en el edificio B de la Facultad de Física, donde por razones estructurales existen varias columnas en sus plantas y en ocasiones aparece un leve tráfico de personas.

\subsection{Objetivos}
En este trabajo se trata de evaluar la capacidad de la tecnología Ultra-wide Band para las tareas de posicionamiento local.
Para este acometido es fundamental tener en cuenta el entorno en el que se intenta conseguir dicho posicionamiento y la disposición de todo el entramado de sensores para encontrar la configuración óptima.

Así, los objetivos de este trabajo son
\begin{itemize}
    \item Evaluar la precisión de los sensores de Ultra-wide Band en un entorno favorable, de tal manera que es posible encontrar los límites de una forma fácilmente parametrizable y sin incluir fuentes adicionales de error.
    \item Evaluar de nuevo la precisión de los sensores en un segundo escenario menos favorable, con obstáculos que puedan interferir en las tareas de posicionamiento y así poder comprobar cómo afectan estas nuevas variables a los errores en la posición obtenida.
    \item Comparar el desempeño de la tecnología en ambos escenarios, y las discrepancias frente a los valores teóricos.
    \item En ambos casos, realizar medidas con distintas configuraciones de los sensores para evaluar cuáles de ellas arrojan mejores resultados y así encontrar el mejor equilibrio entre su desempeño y su complejidad.
    \item El aprendizaje y familiarización con el funcionamiento de un robot autónomo y su desplazamiento en entornos controlados.
\end{itemize}