El éxito de los sistemas de posicionamiento global como el GPS ha iniciado en los últimos años el desarrollo de tecnologías y experiencias basadas en el control y el seguimiento de la posición de objetos o personas.

En ambientes exteriores y abiertos, la precisión de este tipo de sistemas, con el equipamiento adecuado, puede llegar a ser de centímetros.
No es el caso en situaciones interiores, donde la nula línea de visión directa entre los medidores y los satélites hacen que esta precisión sea imposible, por lo que su uso, en general, es inviable.

Debido a estos inconvenientes aparecen nuevas técnicas para conseguir un posicionamiento similar, pero de forma local.
Con el mismo resultado como objetivo estas tecnologías tienen serias diferencias debido a las circunstancias para las que se desarrollan: entre ellas se encuentran principalmente la precisión requerida, mucho mayor en interiores, o la necesidad de lidiar con un espectro electromagnético más poblado.

De la misma forma que en el caso del posicionamiento global, los algoritmos usados se basan principalmente en argumentos geométricos a partir de emisores con posición conocida, y es la tecnología a partir de las que se determinan los parámetros necesarios la principal diferencia entre las implementaciones comerciales disponibles.

La proliferación de la comunicación inalámbrica ha provocado una saturación en el espectro electromagnético, de tal forma que nuevos usos que puedan emplear este tipo de ondas se vean seriamente afectados por interferencias en la frecuencias en uso.

Aunque su planteamiento se remonta varias decenas de años en el tiempo, es cada vez más habitual la mención al Ultra-wide Band (UWB), una tecnología que usa un espectro de ondas de radio con un ancho de banda superior al 20\% de la frecuencia central o mayor que 500 MHz.

Su gran ancho de banda permite la emisión de pulsos muy cortos en el tiempo y de baja energía, lo que hace que su implantación tenga una alta eficiencia.
Además no requiere diseños complejos para la emisión ni para la recepción de pulsos, con lo que el equipamiento necesario para su uso tendrá en general un bajo coste.

La baja tolerancia que proporciona un uso tan amplio de frecuencias a las interferencias --a diferencia del Bluetooth o Wi-Fi, con canales con ancho de banda mucho más pequeño-- la hacen muy atractiva para su uso masivo teniendo en cuenta que es posible asegurar un buen funcionamiento independientemente del uso simultáneo de otras tecnologías.

El estudio de la precisión disponible del UWB es el principal foco de estudio de este trabajo.
La baja potencia de los pulsos usados para el posicionamiento hace que la presencia de obstáculos sea su principal punto débil, con una capacidad de penetración muy baja.

En escenarios con una configuración en la que el objetivo del posicionamiento no tenga línea visión directa de las balizas no será posible tener mediciones excesivamente precisas, y determinar los límites en dichos casos será crítico para evaluar el uso del UWB o de alguna de sus alternativas.

Para automatizar la toma de datos en la medida de lo posible se ha usado un robot en el que colocar el sensor de UWB a posicionar, de tal forma que ha sido posible una toma de datos con un volumen mayor a comparación de una metodología manual.

El robot utilizado ha sido el Turtlebot 2, un pequeño robot con fines educativos que permite el uso de cámaras y sensores láser para labores de movimiento autónomo, de tal forma que permite un movimiento total en un entorno controlado.
Aunque es posible una navegación libre a partir de los sensores mencionados, para este trabajo se creó un mapa de los dos escenarios en los que se produjo la toma de medidas para que su posicionamiento local fuese lo más preciso posible.

Dicho robot fue empleado en dos escenarios para poder comparar el desempeño de los sensores de UWB en dos situaciones distintas.
Así, en la elección los dos escenarios se primó la posibilidad de comparar una situación ideal, con visión directa y distancias cortas; y una situación más parecida a un uso habitual, con un escenario en el que, en ocasiones, pueda haber obstáculos que dificulten el traslado de la señal de UWB.

La primera toma de medidas se llevó a cabo en uno de los laboratorios de los Institutos Universitarios de Investigación en el que se disponía de una superficie amplia y libre; la segunda localización tuvo lugar en el edificio B de la Facultad de Física, donde por razones estructurales existen varias columnas en sus plantas y en ocasiones aparece un leve tráfico de personas.